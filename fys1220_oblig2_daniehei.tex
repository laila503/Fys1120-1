\documentclass[a4paper,norsk, 10pt]{article}
\usepackage[utf8]{inputenc}
\usepackage{verbatim}
\usepackage{listings}
\usepackage{graphicx}
\usepackage[norsk]{babel}
\usepackage{a4wide}
\usepackage{color}
\usepackage{amsmath}
\usepackage{float}
\usepackage{amssymb}
\usepackage[dvips]{epsfig}
\usepackage[toc,page]{appendix}
\usepackage[T1]{fontenc}
\usepackage{cite} % [2,3,4] --> [2--4]
\usepackage{shadow}
\usepackage{hyperref}
\usepackage{titling}
\usepackage{marvosym }
\usepackage{subcaption}
\usepackage[noabbrev]{cleveref}


\setlength{\droptitle}{-10em}   % This is your set screw

\setcounter{tocdepth}{2}

\lstset{language=c++}
\lstset{alsolanguage=[90]Fortran}
\lstset{alsolanguage=Python}
\lstset{basicstyle=\small}
\lstset{backgroundcolor=\color{white}}
\lstset{frame=single}
\lstset{stringstyle=\ttfamily}
\lstset{keywordstyle=\color{red}\bfseries}
\lstset{commentstyle=\itshape\color{blue}}
\lstset{showspaces=false}
\lstset{showstringspaces=false}
\lstset{showtabs=false}
\lstset{breaklines}
\title{FYS1120 Oblig 2}
\author{Daniel Heinesen}
\begin{document}
\maketitle

\paragraph*{Oppgave 2}
\subparagraph*{a)}
Med et uniformt magnetfelt vil kraften

\begin{equation}
\vec{F} = q \vec{v}\times \vec{B}
\end{equation}

alltid stå vikelrett på hastighetsvektoren, hvilket betyr at vi kan forvente en sikulærbane. Vi vil senere i oppgaven løse likningen for denne bevegelsen analytisk.\\

Programmet gir disse plottene:

\begin{figure}[H]
\begin{center}
\includegraphics[width = 70mm]{opp2PosKomp2d.png}
\caption{$x$, $y$ og $z$ plottet mot tid.}
\end{center}
\end{figure}

Vi kan se at $x$ og $y$ varierer som en $cosinus$- og en $sinus$funksjon -- $x$ er noe forskyvet --, mens $z$ holder seg konstant på 0.

\begin{figure}[H]
\begin{center}
\includegraphics[width = 70mm]{opp2VelKomp2d.png}
\caption{$v_x$, $v_y$ og $v_z$ plottet mot tid.}
\end{center}
\end{figure}

Som med posisjonene er $z = 0$, og $x$ og $y$ varierer er $cosinus$- og en $sinus$funksjoner.

\begin{figure}[H]
\begin{center}
\includegraphics[width = 70mm]{opp2Pos3d.png}
\caption{Vi kan se at elektronet beveger seg i en sirkelbane, som vi forventet.}
\end{center}
\end{figure}

Vi kan se i figurene over at elektronet beveger seg i en sirkelbane, det var det vi forventet fra det lille resonnomentet i starten av oppgaven.

\subparagraph*{b)}
For å finne omløpstiden til numerisk kan vi bruke 2 metoder. Den visuelle og unøyaktige er å plotte $|\vec{r}|$ mot $t$:

\begin{figure}[H]
\begin{center}
\includegraphics[width = 70mm]{opp2R2d.png}
\caption{Avstanden til origo over tid.}
\end{center}
\end{figure}

Vi kan finne omløpstiden ved å se når avstanden fra origo er 0. Vi kan se avstanden er 0 ca mellom $1.7$ og $1.8\cdot 10^{-11} $. En mer nøyaktig måte får å finne omløpstiden er å sjekke når vinkelen til elektronet i forhold til $x-aksen$. Omløpstiden vi får fra dette er $1.8021\cdot 10^{-11}$ s. Med det analytiske utrykket vi skal utlede om noen deloppgaver, finner vi en verdi på $1.789 \cdot 10^{-11} s$. Vi ser at det numeriske svaret er veldig likt det analytiske.

\newpage

\subparagraph*{c)}
For å finne syklotronfrekvesen ser vi at for å gå i en sirkerbane må størrelsen til kraften fra magnetfeltet være lik sentripitalkraften.

$$
F_{magnet} = F_{sentripital}
$$

\begin{equation}
qvB = m\frac{v^2}{r}
\end{equation}


\begin{equation}
R = \frac{vm}{qB}
\end{equation}

Vi bruker sammenhengen mellom frekvensen og radius:

\begin{equation}
\omega_s = \frac{v}{R} = \frac{vqB}{vm} = \frac{qB}{m}
\end{equation}


Perioden er tiden det tar å komme seg $2\pi$ rundt sirkelen med vinkelhastighet $\omega$

\begin{equation}
T = \frac{2\pi}{\omega} = \frac{2\pi m}{qB} = 1.789\cdot 10^{-11} s
\end{equation} 

\subparagraph*{d)}

Har elektronet en hastighet i z-retning, vil vi få en bane som skrur oppover.
Jeg bruker den analytiske løsningen, og plotten den over den numeriske. Den analytisk løsningen er gitt ved:

\begin{equation}
\begin{split}
r_x(t) = \frac{v_{x0}}{\omega}(1-\cos(\omega t))
\\
r_y(t) = -\frac{v_{x0}}{\omega}\sin(\omega t)
\end{split}
\end{equation}

For $x-$ og $y-retningen$ (Vises i Appendix A), og for $z-retning$

\begin{equation}
r_z(t) = v_{z}*t
\end{equation}

Plottet blir da:

\begin{figure}[H]
\begin{center}
\includegraphics[width = 70mm]{opp2dPos3dwAnalytic.png}
\caption{Posisjonen til et elektron med en fart i $z-retning$.}
\end{center}
\end{figure}

En oppover skruebane. Vi kan se at den analytiske og numeriske løsningen er (mer eller mindre) helt like.


\paragraph*{Oppgave 3}
\subparagraph*{a)}
I mitt program bruker jeg et magnet felt på $B = 2T$. Jeg får da:

\begin{figure}[H]
\begin{center}
\includegraphics[width = 70mm]{opp3aPos.png}
\caption{En sirkelbane med økende radius.}
\end{center}
\end{figure}

Protonet akselereres når det går gjennom det elektriske feltet i $the valley gap$, og bøyes i det magnetiske feltet. E-feltet varierer slik at protonen alltid akselereres i hastighetsretningen. Dette resulterer i en sirkelbevegelse hvor radiusen øker for hver gang protonet går gjennom $valley gapet$

\subparagraph*{b)}

Vi lar nå protonet slippes løs fra syklotronen når avstanden fra sentrum er $90 \mu m$ får vi:

\begin{figure}[H]
\begin{center}
\includegraphics[width = 70mm]{opp3bPosKomp.png}
\caption{Plott av $x$, $y$ og $z$ over tid.}
\end{center}
\end{figure}

Som i oppgave 2 får vi $cosinus-$ og $sinusfunksjoner$ for $x$ og $y$, og en konstant $z$. Siden radiusen øker vil amplitudene $cosinus-$ og $sinusfunksjonen$ øke. Etter $90 \mu m$ slipper protonet løs og fortsetter i den retningen den hadde.

\begin{figure}[H]
\begin{center}
\includegraphics[width = 70mm]{opp3bVelKomp.png}
\caption{Plott av $v_x$, $v_y$ og $v_z$ over tid.}
\end{center}
\end{figure}

$v_x$, $v_y$ og $v_z$ likner mye på posisjonskomponentene. Vi kan se at $v_x$ plutselig øker nær topp- og bunnpunktene. Dette er når protonet akelsereres i E-feltet. På slutten slippes protonet løs og protonet opplever ingen akselerasjon og hastigheten forblir konstant.


\subparagraph*{c)}

Her er et plot av farten over tid:

\begin{figure}[H]
\begin{center}
\includegraphics[width = 70mm]{opp3bSpeed.png}
\caption{Farten til protonet som en faktor av lyshastigheten $c$.}
\end{center}
\end{figure}

Vi kan se farten øker brått når protonet går gjennom E-feltet, for så å bli konstant når det bare påvirkes av magnetfeltet. Vi kan se at mot sluttet blir farten konstant. Tar vi å sjekker lengden av hastighetsvektoren i siste tidspunkt finner vi unnslipsfarten:

$$
|\vec{v}_{last}| = 8918994.96 m/s = 0.00297 c
$$

\subparagraph*{d)}

Vi bruker eq 1 \textbf{Sjekk nummeret på likningen} og løser uttrykket for v, og får:
\begin{equation}
v = \frac{qBr}{m} 
\end{equation}

Vi setter nå dette uttrykke som hastigheten i den kinetiske energien:

\begin{equation}
E = \frac{1}{2}mv^2 = \frac{1}{2} \frac{q^2B^2r^2}{m}
\end{equation}

\paragraph*{Oppgave 4}

\paragraph*{Kode:}
\hspace{5mm}

\lstinputlisting{oppgave3b.py}


\paragraph*{Appedix A: Utreging av analytisk løsning.}
\hspace{5mm}

Vi skal vise at vi får en sirkelbane i oppgave 2, ved å løse problemet analytisk:\\

Vi starter med at

\begin{equation}
\vec{a} = \frac{q}{m}(\vec{v} \times \vec{B})
\end{equation}

Vi setter det så opp komponentvis, siden vi bare har et magnetfelt i z-retning, setter vi $B_z = B$:

$$
\vec{v} \times \vec{B} = \hat{i}(v_yB) - \hat{j}(v_xB)
$$

\begin{equation}
\begin{split}
a_x = v_x' = - \frac{qB}{m}v_y
\\
a_y = v_y' =  \frac{qB}{m}v_x
\end{split}
\end{equation}

vi bruker at $\frac{qB}{m} = \omega$(vises i en senere deloppgave). Den generelle løsningen på denne differensiallikningen er:

\begin{equation}
\begin{split}
v_x(t) = c_1\sin(\omega t) + c_2\cos(\omega t)
\\
v_y(t) = c_1\cos(\omega t) - c_2\sin(\omega t)
\end{split}
\end{equation}

setter vi inn grensebetingelsene $v_x(0) = v_{x0}$ og $v_y(0) = 0$ får vi 

\begin{equation}
\begin{split}
v_x(t) = v_{x0}\cos(\omega t)
\\
v_y(t) = -v_{x0}\sin(\omega t)
\end{split}
\end{equation}

Om vi integrerer dette og setter inn at $r_x(0) = r_y(0) = 0$ får vi et uttrykk for posisjonen:

\begin{equation}
\begin{split}
r_x(t) = \frac{v_{x0}}{\omega}(1-\cos(\omega t))
\\
r_y(t) = -\frac{v_{x0}}{\omega}\sin(\omega t)
\end{split}
\end{equation}

Dette gir oss en sirkelbevegelse hvor x-komponenten er litt forskyvet, akkurat som vi så i den første figuren. Radiusen på sirkelen er $R = \frac{v_{x0}}{\omega}$. 



\end{document}


